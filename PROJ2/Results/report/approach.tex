\subsection{Variáveis de Decisão}

Tal como descrito na secção anterior, as variáveis de decisão serão os números de cada "uva".
De modo, a representar um \textit{puzzle} como o da Fig.\ref{fig: 4rowproblem}, optou-se por uma lista de listas, em que cada lista representa uma linha do problema.

O predicado \verb|defineDomains/1| é responsável por definir os domínios para cada linha. Inicialmente, atribui à primeira linha o domínio \verb|[1,9]| e depois vai atribuindo às outras listas o domínio \verb|[2,MaxValue]|. A variável \textbf{MaxValue} é calculada consoante o número de linhas.

\subsection{Restrições}

As restrições deste problema são as seguintes:
\begin{itemize}
    \item Primeira linha apenas pode conter números positivos de um dígito
    \item A "uva" que se encontra debaixo de duas "uvas" é a soma destas
    \item "Uvas" com a mesma cor, contêm o mesmo número, com excepção da cor branca
    \item Há um número máximo de cores
\end{itemize}

As restrições rígidas do problema são as três primeiras e a última é flexível, pois para problemas com apenas 2 linhas, por exemplo, não seria possível ter 4 cores repetidas.

Assim, para a primeira restrição fez-se uso do predicado \verb|domain/3| da biblioteca \textbf{clpfd} do SICStus para limitar o domínio, como referido na subsecção anterior.
Para a restrição da soma, desenvolveu-se o predicado \verb|defineSumConstraints/1|, que percorre as listas e coloca a restrição \verb|FirstUpper + SecondUpper #= Child|.
Por fim, a restrição de cor é feita com recurso ao predicado \verb|global_cardinality/2|. Este predicado permite contar o número de ocorrências de números numa lista e com uma segunda chamada deste predicado, é possível restringir a ocorrência de pares de números ao número de cores necessárias.

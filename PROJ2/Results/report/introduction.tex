\section{Introdução}
Este artigo descreve o segundo trabalho realizado para a unidade curricular de Programação em Lógica, tendo como objetivo a resolução e geração de problemas do tipo \textit{Grape Puzzles}, através de programação em lógica com restrições.

Inicialmente, o objetivo deste trabalho foi desenvolver um predicado que permitisse resolver os problemas apresentados no enunciado disponibilizado~\cite{ref_url1}. Após esta fase inicial, o próximo passo foi o melhoramento do nosso solucionador de modo a permitir a geração de problemas.

O artigo começa por descrever o problema em questão, seguido da abordagem tomada para o resolver. Após isto, são apresentadas as formas de visualização dos problemas e das soluções e também são apresentadas as experiências e os resultados. Finalmente, são discutidas as conclusões do trabalho.